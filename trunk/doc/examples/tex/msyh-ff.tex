\documentclass[a4paper,12pt]{report}

\usepackage{fontspec} % provides font selecting commands
\usepackage{xunicode} % provides unicode character macros
\usepackage{xltxtra}  % provides some fixes/extras
\usepackage{fullpage}
\usepackage{indentfirst}

% leave hyperref until last
\usepackage[colorlinks=true,bookmarks=true,
    pdftitle={Microsoft YaHei Font},
    pdfsubject={Use Microsoft YaHei Font in XeTex},
    pdfkeywords={YaHei, Font, Microsoft, XeTex, pdf},
    pdfauthor={Dongsheng Song}]{hyperref} 

\pagestyle{empty}

%
% In FontForge, YaHei Bold's 'Fontname' is 'MicrosoftYaHei', should be fixed to 'MicrosoftYaHeiBold'.
% In FontCreator, YaHei Bold's 'postscript name' is 'MicrosoftYaHei', should be fixed to 'MicrosoftYaHeiBold'.
%
\setromanfont{Microsoft YaHei}
\setsansfont{Microsoft YaHei}
\setmonofont{Microsoft YaHei}

\begin{document}

\XeTeXlinebreaklocale "zh"
\XeTeXlinebreakskip = 0pt plus 1pt

\chapter{Microsoft YaHei}

\section{英文}
  \subsection{Regular}
    In the unlikely event that you are employed on a continuing basis to do
    translation, we may need a disclaimer from your employer as well, to assure
    your employer does not claim to own this work.  Please contact the FSF to
    ask for advice if you think this may apply to you.
  \subsection{Bold}
    {\bf In the unlikely event that you are employed on a continuing basis to do
    translation, we may need a disclaimer from your employer as well, to assure
    your employer does not claim to own this work.  Please contact the FSF to
    ask for advice if you think this may apply to you.}

\section{中文}
  \subsection{Regular}
  我们会继续测试 FOP 1.x 的兼容性,希望当 FOP 的下一个版本发布时,
  已经完成对于我们的应用而言的充分测试,从而我可以在 TSVN 的官方
  版本库提交基于 FOP 1.x 的构建代码,淘汰 FOP 0.20.5。
  \subsection{Bold}
  {\bf 我们会继续测试 FOP 1.x 的兼容性,希望当 FOP 的下一个版本发布时,
  已经完成对于我们的应用而言的充分测试,从而我可以在 TSVN 的官方
  版本库提交基于 FOP 1.x 的构建代码,淘汰 FOP 0.20.5。}

\end{document}
