%!TEX TS-program = xelatex
%!TEX encoding = UTF-8 Unicode

\documentclass[a4paper,12pt,openany]{report}

\usepackage{fontspec} % provides font selecting commands
\usepackage{xunicode} % provides unicode character macros
\usepackage{xltxtra}  % provides some fixes/extras
\usepackage{fullpage}
\usepackage{indentfirst}

% leave hyperref until last
\usepackage[colorlinks=true,bookmarks=true,
    pdftitle={Microsoft YaHei Font},
    pdfsubject={Use Microsoft YaHei Font in XeTex},
    pdfkeywords={YaHei, Font, Microsoft, XeTex, pdf},
    pdfauthor={Dongsheng Song}]{hyperref}

\pagestyle{empty}

\setromanfont[BoldFont={[msyhbd.ttf]}]{Microsoft YaHei}
\setsansfont[BoldFont={[msyhbd.ttf]}]{Microsoft YaHei}
\setmonofont[BoldFont={[msyhbd.ttf]}]{Microsoft YaHei}

\begin{document}

\XeTeXlinebreaklocale "zh"
\XeTeXlinebreakskip = 0pt plus 1pt

\chapter{fonts}

\section{Microsoft}
  \subsection{Microsoft YaHei}
    \fontspec{Microsoft YaHei}
  我们会继续测试 FOP 1.x 的兼容性,希望当 FOP 的下一个版本发布时,
  已经完成对于我们的应用而言的充分测试,从而我可以在 TSVN 的官方
  版本库提交基于 FOP 1.x 的构建代码,淘汰 FOP 0.20.5。

  \subsection{Microsoft YaHei Bold}
    \fontspec{Microsoft YaHei Bold}
  {\bf 我们会继续测试 FOP 1.x 的兼容性,希望当 FOP 的下一个版本发布时,
  已经完成对于我们的应用而言的充分测试,从而我可以在 TSVN 的官方
  版本库提交基于 FOP 1.x 的构建代码,淘汰 FOP 0.20.5。}

\end{document}
